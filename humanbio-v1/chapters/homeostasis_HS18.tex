\section{Homeostasis}\label{sec:HomeostaticControl}
		\begin{center}
		 \fboxp{12cm}{The maintencance of the \textbf{internal environment} of a body is considered a relatively stable condition and this is called \texttt{HOMEOSTASIS}.
		\texttt{note: homeostasis is \texttt{dynamic} process, in which internal adjustments are made continuously to compensate for extranal changes.}}
		\end{center}

\subsection{Analogy to explain homeostatic feedback control}		
Figure \ref{fig:HomeostasisAnalogy} summarises a simple experiment to keep the temperature of a basin full of water at a steady temperature:
\vspace{1cm}


		
		\begin{figure*}[htp]
		  \includegraphics[width=1\textwidth]{/share/SB_Unterricht/Biologie/hum14_RegelKreise/Regelkreis-Analogie-Wasser_v1.png}
		  \caption[Homeostasis model analogy with cold and warm water; unknown source, adapted by saz]{Analogy model to explain homeostatic control}
		  \label{fig:HomeostasisAnalogy}
		%\setfloatalignment{b}
		\end{figure*}
\clearpage


\begin{comment}
	\begin{landscape}
\bgroup 
 \large

%  	\begin{table}[]
% 	\setlength{\extrarowheight}{6pt}	
	 \begin{tabularx}{22cm}[]{|p{10cm} | p{10cm}|} %
	\toprule
	\subsection*{student 1} pour a beaker full of \textbf{cold} water into the basin! 
	\newline repeat this upon instruction by the teacher &
	
	\subsection*{student 2} Read the temperature of the thermometer and continuously tell the figures student No 3! \vspace{18pt} \\ \midrule
	
	 \subsection*{student 3} Call student No 4 for action, in order to keep the temperature at the \textbf{initial} value! 
	 \newline $\Rightarrow$ show the \textbf{red} card if the temperature rises above the initial value!   
	 \newline $\Rightarrow$ show the \textbf{blue} card if the temperature falls \textbf{below} the initial value! &
	 
	  \subsection*{student 4} Act according to the instructions given by student No3:    
	  \newline $\Rightarrow$ pour \textbf{cold} water into the basin whenever the \textbf{red} card is shown!  
	   \newline $\Rightarrow$ pour \textbf{hot} water into the basin whenever the \textbf{blue} card is shown!  \vspace{18pt}\\
	\bottomrule
	\end{tabularx}%
% 	\end{table}% 
 \egroup
\end{landscape}
\newpage
... empty page...
\clearpage
 \addtocounter{page}{-2}
\end{comment}

	 \areaset[0cm]{16cm}{27.4cm}  

			\begin{figure}[htp]
		  \includegraphics[width=1.1\textwidth]{/share/SB_Unterricht/Biology/hum14_homeostasis/Homeostasis-Circle-scheme.png}
		  \caption[Homeostasis circular model, according to natura3]{ Homeostatic control works as a circular feedback system.}
		  \label{fig:HomeostasisCircle}
		%\setfloatalignment{b}
		\end{figure}
	
	\vfill
	\hspace{-2cm}
% 	\hspace{-5cm}
% 	\begin{addmargin*}[-3cm]{-5cm}
\begin{minipage}{18cm}
 		 \includegraphics[width=1\textwidth]{/images/Russel-Dynamic_p0878-fig-1_v2.png}
\captionof{figure}[Homeostasis, negative feedback system; Russel (3rd Ed), Fig. 38.9]{Components of a negative feedback control system maintaining homeostasis. The sensor, integrator, and effector(s) are physical components in the body, such as a body structure or chemical.}\label{fig:HomeostasisNegativeFeedback}
\end{minipage}
% 	\end{addmargin*}	
	\vfill
	
% 	 \areaset[0cm]{11.5cm}{27.4cm}  
\begin{landscape}
\subsection{Control of breathing}
Maintencance of an adequate concentration of  \ce{O2} in your blood, the lymph and the interstitial fluid is absolutely crucial. The following scheme (fig. \ref{fig:HomeostasisBreathing}) visualises this homeostatic control mechanism and \ding{229} Starr Chapter 35.5 on p. 620 (9ed) describes this process.

\thispagestyle{empty} 
		\Ersatz{
		\hspace{-1cm}
		\begin{minipage}{22cm}
			\includegraphics[width=1\textwidth]{/share/SB_Unterricht/Biologie/hum14_RegelKreise/Physiol-Malatl_Atmung-i_056-farbig.png}
		  \captionof{figure}[Homeostasis of breathing; according to Malatlas Physiologie]{Maintencance of a homeostasis of the respiratory system is based on the measurement of  \ce{CO2}-concentration in the cerebral fluid.}
		  \label{fig:HomeostasisBreathing}
		\end{minipage}
		}{%%
		\hspace{-1cm}
		\begin{minipage}{22cm}
			\includegraphics[width=1\textwidth]{/images/PhysiolMalatl_056-b1_Atmung-i.png}
		  \captionof{figure}[Homeostasis of breathing; according to Malatlas Physiologie]{Maintencance of a homeostasis of the respiratory system is based on the measurement of  \ce{CO2}-concentration in the cerebral fluid.}
		  \label{fig:HomeostasisBreathing}
		\end{minipage}
		}

		
		
		




% \begin{addmargin*}[0cm]{-4cm}
\enlargethispage{2cm}
\hspace{2cm} \vspace{0.4cm}
\begin{minipage}{18cm}
% 	\captionof{table}[Homeostasis; terms used by Starr, Chapter 35.6 (8ed)]{Terms used by Starr to explain homeostasis of breathing:}
% 	  \vspace{3pt}  \hspace{0cm}
	    \begin{tabularx}{17cm}[]{X X | p{0.3cm} X X} %
	\toprule
	neuron  & Nervenzelle && medulla oblongata& verlängertes Rückenmark  \\
	action potential & Aktionspotential = Nervensignal && cerebrospinal fluid & Liquor = Flüssigkeit in Hirn und Rückenmark \\
	chemoreceptor & Sinnesorgan für einen bestimmten chemischen Stoff && carotid artery & Carotis = Halsschlagader \\
	\bottomrule
	\end{tabularx}%
	  \label{tab:HomeostasisTerms}%
\end{minipage}

\end{landscape}		
\clearpage


% \end{addmargin*}

% 		\piccaptionoutside 					
% 		%\setcapmargin*[-2cm ]{0cm }  
% 		\piccaption[Homeostasis, negative feedback system; Russel (3rd Ed), Fig. 38.9]{\label{fig:HomeostasisNegativeFeedback} Components of a negative feedback control system maintaining homeostasis. The sensor, integrator, and effector(s) are physical components in the body, such as a body structure or chemical.}    
% 		\parpic[r]{\includegraphics[width=16cm]{/images/Russel-Dynamic_p0878-fig-1_v2.png}}
% 		\picskip{0}
% 		%\setcapmargin*[0cm ]{0cm }
% 		..
% 		\vspace{1cm}
% 		
% 				\piccaptionoutside 					
% 		%\setcapmargin*[-2cm ]{0cm }  
% 		\piccaption[Homeostasis, Husky example of a negative feedback system; Russel (3rd Ed), Fig. 38.10]{\label{fig:HomeostasisNegativeFeedback}Homeostatic negative feedback control system maintaining the body temperature of a dog when environmental temperatures are high. }    
% 		\parpic[r]{\includegraphics[width=14cm]{/images/Russel-Dynamic_p0879-fig-1_v2.jpg}}
% 		\picskip{0}
% 		%\setcapmargin*[0cm ]{0cm }
		
		
