

\areaset[0cm]{11.5cm}{27.4cm}
\section{The mammal Heart and its cardiac cycle}

			\begin{mdframed}[style=exampledefault, userdefinedwidth=12cm,frametitle={Starr 33.3}\label{mat:BEISPIELMATERIAL}]
			Starr gives us a nice introduction and a general view on this topic: read \ding{229} prior to our ``hands-on'' heart dissection lesson!
			Additionally, explore the animated heart exploration provided by the British Heart Foundation \Forward ~~\textit{youtube link:} \href{http://www.bhf.org.uk/heart-health/how-your-heart-works/know-your-heart.aspx}{www.bhf.org.uk}
		\end{mdframed}

	\begin{enumerate}[resume,series=chapter]
	\item Read the following text and write the terms both in \emph{English} and \emph{German} next to the coloured figure \ref{fig:HeartOpenTortora}. Do also compare with the overview to the cardiac cycle given in \ref{fig:KreislaufSchemaLeer} on page \pageref{fig:KreislaufSchemaLeer}!
	\end{enumerate}
% \onesidemarginals


\Ersatz{
		\begin{figure}[htp] \centering
		  \includegraphics[width=1\textwidth]{/share/00_SCHULE_DATA-add/bilder_saz/Humanbio_Bilder/humanbio_Tortora-en/ch20/pap14_fig_20_04a.jpg}
		  \caption[Heart cut open (frontal); from Tortora ch 20]{Frontal section of  a human heart.}
		  \label{fig:HeartOpenTortora}
		%\setfloatalignment{b}
		\end{figure}
		}{
		\begin{figure}[htp] \centering
		  \includegraphics[width=0.85\textwidth]{/share/00_SCHULE_DATA-add/bilder_saz/Humanbio_Bilder/humanbio_Tortora-en/ch20/pap14_fig_20_07a_v2.jpg}
		  \caption[Heart cut open (frontal); from Tortora ch 20]{Frontal section of  a human heart.}
		  \label{fig:HeartOpenTortora}
		%\setfloatalignment{b}
		\end{figure}
		}

 The tissue\sidenote{Gewebe} beneath the nail on your big toe is called the nail matrix, or nail bed. The function of the nail bed is to produce new cells (by mitosis) that then transform into nail substance and thus let your toenail grow by roughly 0.1 mm per day. In order to fulfil this function, the nail bed tissue needs nutrients and oxygen ( \ce{O2}).
These substances are transported there in the blood vessels which in turn form part of the circulatory system. Waste products of nail matrix metabolism\sidenote{Stoffwechsel} such as carbon dioxide  (\ce{CO2}) are removed in the blood vessels. And because all living cells require nutrients and oxygen and are dependent on waste removal, every cell in your body is no further than 1 mm away from a blood vessel. The blood vessels from the lower body, e.g. from your big toe, from feet and legs or from the intestines \sidenote{Darm} eventually gather in the main vein from the lower body, the \textit{inferior vena cava}\sidenote{Untere Hohlvene}. The blood in the \textit{vena cava} is low in oxygen and thus is of a dark red colour and the same is true for the blood in the \textit{superìor vena cava}\sidenote{Obere Hohlvene}, the main vein that brings the blood from the upper body.

The \textit{inferior} and \textit{superior} \textit{vena cava} enter the heart in the\textbf{ right atrium}\sidenote{Vorhof}. \emph{Veins} are defined as blood vessels that carry blood to the heart. From the right atrium blood passes through the \textbf{tricuspid valve}\sidenote{Rechte Segelklappe} also known as the right atrio-ventricular valve into the \textbf{right ventricle}\sidenote{Rechte Herzkammer}. As the heart muscle contracts, the blood is pumped out of the right ventricle, through the \textbf{pulmonary valve}, also known as the \textbf{right semi-lunar valve}\sidenote{Rechte Taschenklappe} and into the \textbf{pulmonary artery} \sidenote{Lungenarterie}.  \emph{Arteries} are defined as blood vessels that carry blood from the heart. The pulmonary artery branches to bring the de-oxygenated blood to both the right and the left lung. Gas exchange then takes place on the surface of the \emph{alveoli}\sidenote{Lungenbläschen}: following their concentration gradients,  \ce{CO2} diffuses from the capillaries into an alveolus and  \ce{O2} diffuses from an alveolus into the capillaries. Oxygenated blood has a bright red colour and this blood, now rich in oxygen, is transported in the left and right \textbf{pulmonary veìn}\sidenote{Lungenvene}  into the \textbf{left atrium}\sidenote{Linker Vorhof}.

From the left atrium, the blood passes through the \textbf{bicuspid valve}, also known as the left atrio-ventricular valve\sidenote{Linke Segelklappe, \textit{Bicuspidalklappe}} into the \textbf{left ventricle}\sidenote{Linke Herzkammer, \textit{li Ventrikel}}. The wall that separates the two ventricular chambers is called the interventricular septum\sidenote{Herzscheidewand}. As the heart muscle contracts, the blood is pumped out of the left ventricle, through the \textbf{aortic valve}, also known as the left semi-lunar valve\sidenote{linke Taschenklappe, \textit{Aortenklappe}} and into the \textbf{aorta}\sidenote{Aorta}, the main artery of our body. Some of the blood in the aorta is used to supply the heart itself - through the \textbf{coronary arteries}\sidenote{Herzkranzgefässe}. In the aortic arch\sidenote{Aortenbogen} blood vessels branch off to bring blood to the upper body. One of the vessels in the upper body, the \textbf{carotid artery}\sidenote{Halsschlagader}, brings the blood into the brain, for example to the visual cortex in the \emph{occipital lobe of the cerebrum}\sidenote{Sehzentrum im Hinterhauptslappen des Hirns}: this is where nerve cells process" the visual information coming from your eyes, now that you are reading this text. After passing through the brain, the blood is again de-oxygenated. It is transported via the superior vena cava into the right heart, from there into the lungs where gas exchange takes place, from the lungs back into the left heart, under high pressure into the aorta and perhaps down again to the nail matrix of your big toe. Taken together: we can distinguish the short distance - low pressure \textbf{pulmonary circulation}\sidenote{Lungenkreislauf} powered by the right heart and the long distance - high pressure \textbf{systemic circulation}\sidenote{Körperkreislauf} powered by the left heart.



	\vspace{4pt}
	\begin{minipage}[htbp]{1\columnwidth}
	\centering {\includegraphics[width=1\linewidth]{/images/heart/anatomy/table-heart.png}} \captionof{figure}[VERZEICHNISEINTRAG]{Fill in the table!}  	\label{fig:BILDLABEL}
	\vspace{2pt}
	\end{minipage}

	\clearpage
	\subsection{Heart dissection}
			\begin{enumerate}[label=\textit{(\arabic*)},leftmargin=0em,series=labcounter]
		      \item Hold the pig's heart in front of you, as if you would look onto it in a (human's) chest.
		      \item Distinguish left and right: one of the heart muscles is \textbf{thicker} than the other one, it is the \gap{...left...} ventricle.
		      \item If a tough bag of tissue is found around the heart, it is the \textbf{pericard}: it allows the heart to slide \gap{... without friction ...}.
		      \item Cut away the pericard using a scalpel, but leave both aorta and pulmonary artery in place!
		      \item Separate these aorta and pulmonary artery: cut the connective tissue in between.
		      \item Refer to figure \ref{fig:HeartOpenTortora} on page \pageref{fig:HeartOpenTortora} or in \ding{229} Starr p. 574: can you identify all the structures labelled?
		      \item Turn the heart over and have a look from the backside! Compare with your figures: a pig heart shows quite some differences, when looked from the back side - don't get confused!
		      \item Rip the Aorta using a scalpel. What are the small holes along the aorta?
		      \item Keep the heart upright, search for the aortic valve and fill its flaps with some water - now, can you fill the entire trunk of the aorta with more water? \gap{... yes, this should be possible - this prevents a backflow of blood into the heart!...}
		      \item Now fill the trunk of the vena cava and the right atrium with water. Can you fill these structures with water too? \gap{.. no, because water is let in by the AV valve...}
		      \item Enter the probe (rubber tube with a piece of glass) through the right atrium and the right ventricle and observe where your probe leaves the heart: \gap{... out of the pulmonary artery ...}
		      \item Cut the heart open using the scalpell according to your teachers instructions!
		      \item Study the mitral (or left AV) valve, the tricuspidal (or right AV) valve, the pulmonary valve (a semi-lunar valve), the aortic valve (the other semi-lunar valve).
		      \item Further you investigate the septum, the myocard, the papillary muscles and the chordae tendinae.
		      \item Look for the coronary arteries using a piece of wire. The function of the coronary arteries is to \gap{...deliver oxygen rich blood to the heart...}
		\end{enumerate}

% \clearpage
\areaset[0cm]{16cm}{26.5cm}
\subsection{The cardiac cycle}
\enlargethispage{2.6cm} \thispagestyle{empty}

				\begin{mdframed}[style=exampledefault, userdefinedwidth=16cm,frametitle={Starr Chapter 33.3}\label{mat:Starr-33.3_CardiacCycle}]
			Starr gives a short explanation on the \textbf{cardiac cycle}, check \ding{229} paragraph ``setting the pace'' on p. 575. This is part of the syllabus. Figure \ref{fig:CardiacCycleRussell} completes Starr's (too) short text.
		\end{mdframed}

%
\begin{minipage}{16cm} \hspace{-2cm}
		  \includegraphics[width=1\textwidth]{/images/figures/Russel-44_17-txt_v1.jpg}
		  \captionof{figure}[Cardiac Cycle; Russel DynamicScience, chp 44]{These figures explain you the course of a cardiac cycle, from the genesis of an impulse at the AV-node to expulsion of blood from the ventricles. }
		  \label{fig:CardiacCycleRussell}
\end{minipage}

% /share/00_SCHULE_DATA-add/bilder_saz/Humanbio_Bilder/humanbio_Tortora-en/ch20/pap14_fig_20_12.jpg


% \begin{enumerate}[resume, leftmargin=*,series=chapter]
% \item  Check moodle for animations of the cardiac cycle and then colour figure \ref{fig:CardiacCyclePhysiolColBook} accordingly.
% \end{enumerate}

% 		\begin{center}
	\begin{minipage}{16cm}
	\centering
		\includegraphics[width=0.65\textwidth]{/images/PhysiolMalatl_036B_v4.png}
		\captionof{figure}[scheme of a cardiac cycle, Physiologie Malatlas, Wynn Capit ]{The electrical potentials measured give the medical professional valid information about the processes taking place within the heart. \textbf{Check moodle for animations} of the cardiac cycle and then colour figure \ref{fig:CardiacCyclePhysiolColBook} accordingly.}
		\label{fig:CardiacCyclePhysiolColBook}
			\end{minipage}
% 			\end{center}


\includepdf[pages=113-116,clip=true,angle=0,scale=0.9,  pagecommand={\thispagestyle{empty}}, viewport=1.75cm 1.3cm 19.9cm 27.4cm, addtotoc={
     113-116,subsection,3,Revision Circulation,GCP-circulation}]	% last p -entry NO COMMA!
     {/share/00_SCHULE_DATA-add/Buecher_Bio/revision_CGP-bio_GSCE/CGP-Bio-GCSE-A4.pdf}

