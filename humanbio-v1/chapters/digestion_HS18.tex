\areaset[0cm]{11.5cm}{26.5cm}
\section{Digestive System} \label{sec:DigestSyst} 
  We eat up to one ton of food yearly (\url{http://www.livescience.com/18070-food-americans-eat-year-infographic.html}).
	 	\marginline{\bgroup  \ttfamily There is an animation about the human digestive system on  \textsc{moodle} \egroup\\ \centering \includegraphics[width=2cm]{/share/SB_Unterricht/moodle-QR-code.png}. }
 Even if we restrict our diet, lets say to 500 kg, this remains a huge figure, approx. eight times your body weight. How does our digestive system cope with this regular load of food? Find this answer and learn even more about the \textsc{digestive system} with help of  \ding{229} \textsc{Starr} and \Forward \textsc{moodle}!

			\begin{mdframed}[style=exampledefault, userdefinedwidth=12cm,%
				%  rightmargin=2cm,
					 frametitle={Starr 36.2}\label{mat:BEISPIELMATERIAL}]	  
			this section is part of our matura syllabus!
		\end{mdframed}


The food processing begins with ripping and crushing food, using a beak, like birds, or teeths, like other vertebrates. 
	\marginline{fill the gaps with these terms: \textsl{fox; crush; rip;  cow; sharp, pointed teeth; broad teeths}}
		\begin{enumerate}[itemsep=1.5em, leftmargin=*]
		\item  What can we learn from looking at different kind of teeths?\\
			 \emph{broad teeths} like those from a \gap{cow ~~} are suitable to \gap{crush ~~} food \hfill \\
			 \emph{sharp, pointed teeth} like those from a \gap{fox ~~} are suitable to \gap{rip ~~} food \hfill
		\end{enumerate}

Within the mouth food is formed into a ball, a so called \textbf{bolus} by aid of the tongue and saliva. The subsequent step, \emph{swallowing}
		\marginline{the process of swallowing may be observed by X-rays - see \textsc{moodle} for the link to  \hrefL{/share/SB_Unterricht/Biologie/hum01_VerdauungErnaehrung/WegDerNahrung_PUZZLE/Roentgen_Aufnahme-_Schluckakt.mp4}{https://www.dropbox.com/s/fw7rlr5t8pp3ja9/Roentgen_Aufnahme-_Schluckakt.mp4?dl=0}{swallowing in an x-ray}}
	 is a highly complex set of actions, best shown with an animation. Check \textsc{moodle} for  \hrefL{/share/SB_Unterricht/Biologie/hum01_VerdauungErnaehrung/Verdauungstrakt/Kauen_Schlucken_ZS_2010-10-29/Schlucken.swf}{https://dl.dropboxusercontent.com/u/28539935/Humanbio/Schlucken.swf}{an Animation of swallowing}. 
	
\begin{enumerate}[resume, leftmargin=*]
% 		\item  Explain how you can make visible the process of swallowing using x-ray technology! \hrefL{/share/SB_Unterricht/Biologie/hum01_VerdauungErnaehrung/WegDerNahrung_PUZZLE/Roentgen_Aufnahme-_Schluckakt.mp4}{https://dl.dropboxusercontent.com/u/28539935/Humanbio/Roentgen_Aufnahme-_Schluckakt.mp4}{swallowing in an x-ray} \\
% 		\Loesung{The person had to eat Barium rich food which absorbs x-rays passing through the body and leaving a white area on the x-ray film}{2cm}
		
		\item On its way through the alimentary canal, food is split in nutrients which are taken up and waste which is not usable to the body. Study chapter 36.3 - 36.6 in Starr (9ed) and take notes next to the different stations of the alimentary canal shown on the following double-page!
		\end{enumerate}

\emph{Hint:} The digestion of food through enzymatic cleavage will later be discussed in details (see chapter \ref{sec:Enzymes}, page \pageref{sec:Enzymes}).

		\vspace{1cm}
		\begin{mdframed}[style=exampledefault, userdefinedwidth=12cm, frametitle={Starr 36.3 \& 36.4 \& 36.5 \& 36.6}\label{mat:BEISPIELMATERIAL}]	  
			These sections are part of our syllabus: learn and revise them using your text book,  including the \textit{self quiz-, data analysis- and critical thinking-} subsections.
		\end{mdframed}


%% in die Präambel (bzw. Vorlage)
%\global\mdfdefinestyle{RedGray}{%
%linecolor=red,linewidth=1.5pt,%
%leftmargin=1cm,rightmargin=1cm,%
%backgroundcolor=gray!40
%}

\hfill
\begin{mdframed}[style=RedGray]
This is \textbf{self-learning unit:} Work in groups; ask the teacher if you encounter problems or in case of uncertainty what to learn / not to learn.

For the \textbf{correction}s: only FOUR variants of the double-page "`alimentary canal"' will be corrected by the teacher - thus you must work in groups and ensure to spread the teacher's feedback back to all members of your group; due-date to hand in your solution is \textbf{January 15th 2019}.
\end{mdframed}
\hfill

% %  \vspace{2cm}
% %  \begin{enumerate}[resume, leftmargin=*]
% % \item  Our liver takes a central role in our metabolism. Check out figure \ref{fig:LeberzelleNaehrstoffe} together with the corresponding gap text (with numbers (1) to (15)).  You are allowed to use your book and webressources as well, in order to solve the crossword on page \pageref{fig:CrossWordLiver}. Which school taught you best to meet these standards?
% % \end{enumerate}
% %
% % 	 \marginnote{\caution[c][BrickRed][competition!]{Which one of the Manchester Grammar Schhol wins?}}

 \clearpage
    \includepdf[pages=1-2,landscape=false,angle=90,scale=1, addtotoc={
     1,subsection,2,alimentary canal,p1}]	% last p -entry NO COMMA!
     {/share/SB_Unterricht/Biology/hum01_digestion_nutrition/DigestiveSystem-A4} 

\clearpage
\enlargethispage{2cm}
\hspace{-4.6cm}
\karo{16cm}{24cm}
